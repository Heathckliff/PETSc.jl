%\documentclass[letter,12pt]{article}
\documentclass{article}
%\usepackage[margin=0.6in]{geometry}
\usepackage[english]{babel}
\usepackage[utf8]{inputenc}
\usepackage{amsmath}
%\usepackage{gensymb}
\usepackage{graphicx}
\usepackage{caption}
\usepackage[colorinlistoftodos]{todonotes}
\usepackage{color}
\usepackage{minted}
\usepackage{fullpage}
\usepackage{subfigure}
\usepackage{float}
\usepackage{verbatim}
\usepackage{courier}
\usepackage{booktabs}
\usepackage{wrapfig}
\usepackage{placeins}
\usepackage{units}
\newcommand{\ttt}{\texttt}
\newcommand{\die}{\partial}

\usepackage{cite}

%\renewcommand{\citeleft}{\textcolor{red}{[}}
%\renewcommand{\citeright}{\textcolor{red}{]}}

%\usepackage[usenames,dvipsnames]{xcolor}
\usepackage{hyperref}
\hypersetup{
 colorlinks=true,
 citecolor=blue,
 linkcolor=red,
 urlcolor=blue}


\title{Julia-Petsc Interface II}
\begin{comment}
\author{
  Jared Crean\thanks{Graduate Student, Rensselaer Polytechnic Institute}, \
  Katharine Hyatt\thanks{Graduate Student, University of California Santa Barbara}, \ and
  Steven G. Johnson\thanks{Associate Professor, Massachusetts Institute of Technology},
}
\end{comment}

\date{\today}

\begin{document}
\maketitle

\section{Introduction} \label{sec:intro}
This document describes the changes to the Julia interface to the Portable, Extensible Toolkit for Scientific Computation (PETSc) since the previous report.
Several new features are supported, particularly those related to mapping 
and parallel bookkeeping, and additional functionality is added to existing
features.



\section{Features} \label{sec:features}

\subsection{Vectors and Matrices} \label{sec:arrays}
\subsection{Construction}
\subsubsection{Vectors}
An additional \texttt{Vec} constructor has been added:
\begin{minted}{julia}
function Vec{T<:Scalar}(v::Vector{T}; comm::MPI.Comm=MPI.COMM_WORLD)
\end{minted}
\noindent which takes a Julia 1 dimensional array and creates a PETSc VECMPI, without
copying the array.  The array is automatically protected from garbage 
collection until the \texttt{Vec} object is itself garbage collected.
If called in parallel, the vector that each process supplies becomes the local
part of the PETSc vector.

Additionally, the ability to get the array that underlies the local part of
a PETSc \texttt{Vec} is now supported through \texttt{LocalArray} type:

\begin{minted}{julia}
type LocalArray{T <: Scalar} <: AbstractArray{T, 1}
\end{minted}

\noindent which supports all the standard indexing behavior of a standard Julia 
1 dimensional array.  Indexing this array is significantly faster than 
indexing the \texttt{Vec} object because the \texttt{LocalArray} has direct
access to the underlying memory.
The outer constructor for this type is:

\begin{minted}{julia}
function LocalArray{T}(vec::Vec{T})
\end{minted}

\noindent A read only version is also supported:

\begin{minted}{julia}
function LocalArrayRead{T}(vec::Vec{T})
\end{minted}

\noindent The function 
\begin{minted}{julia}
function LocalArrayRestore{T}(varr::Union{LocalArray{T}, LocalArrayRead{T}})
\end{minted}

\noindent should be called when operations on either type of \texttt{LocalArray} are 
completed.

\subsubsection{Matrices}
The PETSc \texttt{SubMatrix} functionality is exposed through the 
\texttt{SubMat} type:
\begin{minted}{julia}
type SubMat{T, MType} <: PetscMat{T}
  p::C.Mat{T}
  assembling::Bool # whether we are in the middle of assemble
  insertmode::C.InsertMode # current mode for setindex!
  data::Any # keep a reference to anything needed for the Mat
end

\end{minted}

\noindent Although the fields and type parameters of this type are identical to the
standard \texttt{Mat} type, the creation of a distinct type facilitates
providing distinct behavior for the \texttt{SubMat}.  All functions which
are provided for \texttt{Mat}s are also provided for \texttt{SubMat}s.
In particular, a \texttt{SubMat} can be indexed using the standard Julia
indexing notation, where the indices are the local indices (rather than global
indices used for \texttt{Mat}s).

A \texttt{Mat} must have a \texttt{LocalToGlobalMapping} registered in order
to create a \texttt{SubMat}.  The outer constructor

\begin{minted}{julia}
function SubMat{T, MType}(mat::Mat{T, MType}, isrow::IS{T}, iscol::IS{T})
\end{minted}

\noindent creates and registers a default \texttt{LocalToGlobalMapping} if needed.  
The default \texttt{LocalToGlobalMapping} relies on the PETSc 
\texttt{MatGetOwnershipRange} function to determine the range of rows
assigned to each process.  This function will return the correct values for all 
\texttt{AIJ} matrices, but may not for the less commonly used matrix formats.
The index sets \texttt{isrow} and \texttt{iscol} determine which rows and
columns of the parent \texttt{Mat} are mapped to the \texttt{SubMat}.

Shell matrices (ie. matrix free linear operators) are now supported by the 
Julia interface.  A shell matrix can be created via:

\begin{minted}{julia}

function MatShell{T}(::Type{T}, mlocal::Integer, nlocal::Integer, ctx::Tuple=();
                  m::Integer=C.PETSC_DECIDE, n::Integer=C.PETSC_DECIDE, 
                  comm=MPI.COMM_WORLD)
\end{minted}

\noindent where the standard arguments determine the size of the matrix object.
The \texttt{ctx} argument is a user supplied tuple that can be retrieved from 
within any callback function defined for the matrix, as a means of passing
additional arguments.  The \texttt{ctx} can be retrieved using the function

\begin{minted}{julia}
function getcontext{T}(mat::Mat{T, C.MATSHELL})
\end{minted}

\noindent which returns the tuple.

Callback functions that perform matrix operations can be registered using 
the function:

\begin{minted}{julia}
function setop!{T}(mat::Mat{T, C.MATSHELL}, op::C.MatOperation, func::Ptr{Void})
\end{minted}

\noindent where \texttt{op} is an enumerated value (available in the PETSc.C module) that specifies what operation the function performs, and
\texttt{func} is a function pointer (typically created using the Julia \texttt{cfunction} function) that has the right signature for the specified function.
Note that the function must take as arguments the low-level versions of all the PETSc
objects (ie. PETSc.C.Mat instead of PETSc.Mat).  These objects can be safely
wrapped inside the high-level objects by calling their inner constructors with
the keyword argument \texttt{first\_instance} set to false.  

For example, 
matrix-vector multiplication for a shell matrix can be defined with the 
function:


\begin{minted}{julia}
function mymult{T}(A::PETSc.C.Mat{T}, x::PETSc.C.Vec, b::PETSc.C.Vec)

  # wrap low level objects into high level ones
  bigA = Mat{T, PETSc.C.MATSHELL}(A, first_instance=false)
  bigx = Vec{T, PETSc.C.VECMPI}(x, first_instance=false)
  bigb = Vec{T, PETSc.C.VECMPI}(b, first_instance=false)

  # retrieve the user supplied tuple
  ctx = getcontext(bigA)

  # do matrix-vector multiplication here
  

  return PetscErrorCode(0)
end
\end{minted}


\subsection{Operations}
Several additional functions are supported for PETSc vectors and matrices.
For \texttt{Vecs}, the function \texttt{localpart} returns a Julia \texttt{Range} object that describes the range of global indices that are owned by the 
local process.
For \texttt{Mats}, the analogous function is \texttt{localranges}, which 
returns 2 \texttt{Range} objects describing the ownership of rows and columns
of the matrix.  Note that this function has the same assumptions as the PETSc
\texttt{MatGetOwnershipRange} described above.
The function \texttt{localIS} creates an index set that provides the same 
information.

The function \texttt{local\_to\_global\_mapping} returns two 
\texttt{ISLocalToGlobalMapping}s that describe the same information for 
the rows and columns of \texttt{Mat}s.  They can be registered with the
\texttt{Mat} using the function \texttt{set\_local\_to\_global\_mapping}.
\texttt{has\_local\_to\_global\_mapping} returns a boolean value indicating
whether a \texttt{Mat} already has a \texttt{LocalToGlobalMapping} registered.

\subsection{Mappings and Index Sets} \label{sec:is}
The PETSc \texttt{ISLocalToGlobalMapping} is now supported and exposes a 
1-based interface.
They can be created using:

\begin{minted}{julia}
function ISLocalToGlobalMapping{T, I <: Integer}(::Type{T}, 
                                indices::AbstractArray{I}, bs=1; 
                                comm=MPI_COMM_WORLD, 
                                copymode=C.PETSC_COPY_VALUES)
\end{minted}

\noindent where the indices are 1-based and \texttt{bs} indicates the block size of the 
index set.  Currently only the default \texttt{copymode} is supported,
but additional modes will be supported in the future.

Alternatively, an existing index set can be used:
\begin{minted}{julia}
function ISLocalToGlobalMapping{T}(is::IS{T})
\end{minted}

\hfill \break

The PETSc ApplicationOrdering (\texttt{AO}) is supported as well.
They can be created using:

\begin{minted}{julia}
function AO{T, I1 <: Integer, I2 <: Integer}(::Type{T}, 
            app_idx::AbstractArray{I1, 1}, petsc_idx::AbstractArray{I2, 1}; 
            comm=MPI.COMM_WORLD, basic=true )
\end{minted}

\noindent where \texttt{app\_idx} is the application indices and 
\texttt{petsc\_idx} are
the PETSc indices the application indices are mapped to.  Both sets of 
indices are 1-based.  The \texttt{basic} is determines the internal
representation of the mapping.  If \texttt{basic} is true, then
the mapping must be one-to-one and onto, otherwise \texttt{basic} must
be false, although this results in poorer performance.

Alternatively, index sets can be used to create the \texttt{AO}
\begin{minted}{julia}
function AO{T}( app_idx::IS{T}, petsc_idx::IS{T}; basic=true )
\end{minted}

\noindent An array of application indices can be transformed to PETSc indices using
the function

\begin{minted}{julia}
function map_app_to_petsc!(ao::AO, idx::AbstractArray)
\end{minted}

\noindent or the inverse transformation can be applied with

\begin{minted}{julia}
function map_petsc_to_app!(ao::AO, idx::AbstractArray)
\end{minted}

\noindent The same operations can b e done on index sets:

\begin{minted}{julia}
function map_app_to_petsc!(ao::AO, is::IS)
function map_petsc_to_app!(ao::AO, is::IS)
\end{minted}



\subsection{Linear Solvers} \label{sec:ksp}
In addition to the standard matrix solve described in the previous report, 
transposed solves are supported as well:

\begin{minted}{julia}
function KSPSolveTranspose!{T}(ksp::KSP{T}, b::Vec{T}, x::Vec{T})
\end{minted}

\noindent where the \texttt{KSP} object is constructed as described previously and 
contains all the options that specify how to perform the solve.  This
allows multiple solves (with different right hand sides) to be performed 
with a single \texttt{KSP} context.  There is also a version that allocates 
the output vector:

\begin{minted}{julia}
function KSPSolveTranspose{T}(ksp::KSP{T}, b::Vec{T})
\end{minted}

Note that shell matrices can be used for KSP solves if the appropriate 
matrix-vector multiplication function is registered with the matrix.

\subsection{Miscellaneous}
Several improvements to the interface have been made that do not neatly fit 
into one of the above categories.
First, the automatically generated wrappers of the C interface have been 
been improved.  Certain C types are more accurately mapped to their Julia 
counterparts, and additional enumerated values that were not previously
captured are now correctly generated.

Second, several functions that are part of the PETSc C interface have had
more usable Julia definitions created, taking advantage of the fact that Julia
arrays know their own sizes:

\begin{minted}{julia}
function SetValues{T}(vec::C.Vec{T},idx::AbstractVector{PetscInt},
                                vals::AbstractVector{T},
                                flag::Integer=INSERT_VALUES)

function GetValues{T}(vec::C.Vec{T}, idx::AbstractArray{PetscInt,1}, 
                             y::AbstractArray{T,1})
function SetValues{ST}(vec::C.Mat,idi::AbstractArray{PetscInt},
                       idj::AbstractArray{PetscInt}, array::AbstractArray{ST},
                       flag::Integer=INSERT_VALUES)

function SetValuesBlocked{ST}(mat::C.Mat, idi::AbstractArray{PetscInt}, 
                          idj::AbstractArray{PetscInt}, v::AbstractArray{ST}, 
                          flag::Integer=INSERT_VALUES)

function GetValues{ST}(obj::C.Mat, idxm::AbstractArray{PetscInt, 1}, 
                   idxn::AbstractArray{PetscInt, 1}, v::AbstractArray{ST})
\end{minted}

\noindent These functions are much more restrictive than those in the high-level
interface, but they never make temporary copies, which are sometimes 
necessary in the high-level interface.

\end{document}







